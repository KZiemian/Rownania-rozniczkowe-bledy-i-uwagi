% Autor: Kamil Ziemian

% --------------------------------------------------------------------
% Podstawowe ustawienia i pakiety
% --------------------------------------------------------------------
\RequirePackage[l2tabu, orthodox]{nag} % Wykrywa przestarzałe i niewłaściwe
% sposoby używania LaTeXa. Więcej jest w l2tabu English version.
\documentclass[a4paper,11pt]{article}
% {rozmiar papieru, rozmiar fontu}[klasa dokumentu]
\usepackage[MeX]{polski} % Polonizacja LaTeXa, bez niej będzie pracował
% w języku angielskim.
\usepackage[utf8]{inputenc} % Włączenie kodowania UTF-8, co daje dostęp
% do polskich znaków.
\usepackage{lmodern} % Wprowadza fonty Latin Modern.
\usepackage[T1]{fontenc} % Potrzebne do używania fontów Latin Modern.



% ----------------------------
% Podstawowe pakiety (niezwiązane z ustawieniami języka)
% ----------------------------
\usepackage{microtype} % Twierdzi, że poprawi rozmiar odstępów w tekście.
\usepackage{graphicx} % Wprowadza bardzo potrzebne komendy do wstawiania
% grafiki.
\usepackage{verbatim} % Poprawia otoczenie VERBATIME.
\usepackage{textcomp} % Dodaje takie symbole jak stopnie Celsiusa,
% wprowadzane bezpośrednio w tekście.
\usepackage{vmargin} % Pozwala na prostą kontrolę rozmiaru marginesów,
% za pomocą komend poniżej. Rozmiar odstępów jest mierzony w calach.
% ----------------------------
% MARGINS
% ----------------------------
\setmarginsrb
{ 0.7in} % left margin
{ 0.6in} % top margin
{ 0.7in} % right margin
{ 0.8in} % bottom margin
{  20pt} % head height
{0.25in} % head sep
{   9pt} % foot height
{ 0.3in} % foot sep



% ------------------------------
% Często przydatne pakiety
% ------------------------------
% \usepackage{csquotes} % Pozwala w prosty sposób wstawiać cytaty do tekstu.
\usepackage{xcolor} % Pozwala używać kolorowych czcionek (zapewne dużo
% więcej, ale ja nie potrafię nic o tym powiedzieć).



% ------------------------------
% Pakiety do tekstów z nauk przyrodniczych
% ------------------------------
\let\lll\undefined % Amsmath gryzie się z pakietami do języka
% polskiego, bo oba definiują komendę \lll. Aby rozwiązać ten problem
% oddefiniowuję tę komendę, ale może tym samym pozbywam się dużego Ł.
\usepackage[intlimits]{amsmath} % Podstawowe wsparcie od American
% Mathematical Society (w skrócie AMS)
\usepackage{amsfonts, amssymb, amscd, amsthm} % Dalsze wsparcie od AMS
% \usepackage{siunitx} % Do prostszego pisania jednostek fizycznych
% \usepackage{upgreek} % Ładniejsze greckie litery
% Przykładowa składnia: pi = \uppi
\usepackage{slashed} % Pozwala w prosty sposób pisać slash Feynmana.
\usepackage{calrsfs} % Zmienia czcionkę kaligraficzną w \mathcal
% na ładniejszą. Może w innych miejscach robi to samo, ale o tym nic
% nie wiem.



% ----------
% Tworzenie otoczeń "Twierdzenie", "Definicja", "Lemat", etc.
% ----------
\newtheorem{twr}{Twierdzenie} % Komenda wprowadzająca otoczenie
% ,,twr'' do pisania twierdzeń matematycznych
\newtheorem{defin}{Definicja} % Analogicznie jak powyżej
\newtheorem{wni}{Wniosek}



% ----------------------------
% Pakiety napisane przez użytkownika.
% Mają być w tym samym katalogu to ten plik .tex
% ----------------------------
\usepackage{ODE} % Pakiet napisany między innymi dla tego pliku.
\usepackage{latexshortcuts}
\usepackage{mathshortcuts}




% --------------------------------------------------------------------
% Dodatkowe ustawienia dla języka polskiego
% --------------------------------------------------------------------
\renewcommand{\thesection}{\arabic{section}.}
% Kropki po numerach rozdziału (polski zwyczaj topograficzny)
\renewcommand{\thesubsection}{\thesection\arabic{subsection}}
% Brak kropki po numerach podrozdziału



% ----------------------------
% Ustawienia różnych parametrów tekstu
% ----------------------------
\renewcommand{\arraystretch}{1.2} % Ustawienie szerokości odstępów między
% wierszami w tabelach.



% ----------------------------
% Pakiet "hyperref"
% Polecano by umieszczać go na końcu preambuły.
% ----------------------------
\usepackage{hyperref} % Pozwala tworzyć hiperlinki i zamienia odwołania
% do bibliografii na hiperlinki.





% --------------------------------------------------------------------
% Tytuł, autor, data
\title{Równania różniczkowe zwyczajne --~błędy i~uwagi}

% \author{}
% \date{}
% --------------------------------------------------------------------





% ####################################################################
\begin{document}
% ####################################################################



% ######################################
\maketitle % Tytuł całego tekstu
% ######################################



% ####################
\Work{ % Autor i tytuł dzieła
  Władimir Igoriewicz Arnold \\
  ,,Równania różniczkowe zwyczajne'',
  \cite{ArnoldRownaniaRozniczkoweZwyczajne1975} }


\CenterTB{Błędy}
\begin{center}
  \begin{tabular}{|c|c|c|c|c|}
    \hline
    & \multicolumn{2}{c|}{} & & \\
    Strona & \multicolumn{2}{c|}{Wiersz} & Jest
                              & Powinno być \\ \cline{2-3}
    & Od góry & Od dołu & & \\
    \hline
    5   & &  7 & 1968 - 196 & 1968 - 1969 \\
    11  & 17 & & mechanice klasycznej & mechanice kwantowej \\
    15  & & 16 & rozdziale 6 & rozdziale 5 \\
    28  & 15 & & wzór (8) & wzór \\
    34  &  9 & & $\dot{ x }_{ 1 } = x_{ 2 }$ & $\dot{ x }_{ 1 } = x_{ 1 }$ \\
    47  & & 12 & $x_{ i } = \varphi_{ i }( x_{ 1 }, \ldots, x_{ n } )$
           & $x_{ i } = \varphi_{ i }( y_{ 1 }, \ldots, y_{ n } )$ \\
    53  & & 14 & obrót & obrót krzywych całkowych \\
    56  & 12 & & osobliwym & nieosobliwym \\
    61  &  7 & & $\xbf$, $\boldsymbol{\alpha}_{ 0 }$
           & $\xbf$, $\boldsymbol{\alpha}$ \\
    64  & & 11 & $\gbf( t_{ 2 }, t_{ 1 }, \xbf )
                = \gbf^{ t_{ 2 } }_{ t_{ 1 } }( \xbf, t_{ 1 } )$
           & $\gbf^{ t_{ 2 } }_{ t_{ 1 } }( \xbf, t_{ 1 } )
             = \gbf( t_{ 2 }, t_{ 1 }, \xbf )$ \\
    64  & & 10 & $( \vpb( t ), t )$ & $( t, \vpb( t ) )$ \\
    66  & 17 & & $\vbf( t, \xbf, \dot{ \alb } )$
           & $\vbf( t, \xbf, \alb )$ \\
    70  & & 13 & $\pdot_{ i } = \pd{ }{ H }{ q_{ i } }$
           & $\pdot_{ i } = -\pd{ }{ H }{ q_{ i } }$ \\
    71  & & 15 & $\frac{ \partial \vbf_{ 0 } }{ \xbf }$
           & $\pd{}{ \vbf_{ 0 } }{ \xbf }$ \\
    72  &  4 & & ,,niezaburzonego'' & ,,zaburzonego'' \\
    73  &  5 & & $\xbf( 0$ & $\xbf( 0 )$ \\
    90  & &  3 & \emph{wraz z pochodną dla} $x = 0$ & \emph{dla} $x = 0$ \\
    92  &  3 & & $U( x( O ) )$ & $U( x( 0 ) )$ \\
    123 &  6 & & $^{ \Rb }A : \Cb^{ m } \to { }^{ \Rb }\Cb^{ n }$
           & ${ }^{ \Rb }A : { }^{ \Rb }\Cb^{ m } \to { }^{ \Rb }\Cb^{ n }$ \\
    125 &  5 & & $\mathbf{I}$ & $I$ \\
    % & & & & \\
    % & & & & \\
    % & & & & \\
    % & & & & \\
    \hline
  \end{tabular}
\end{center}
\noi
\StrWd{66}{7} \\
\Jest \emph{tyłu do~brzegu} \\
\Powin  \emph{tyłu nieograniczenie albo~do~brzegu} \\
\StrWg{110}{9} \\
\Jest sumą częściową szeregu --~iloczynu \\
\Powin  jest sumą części wyrazów iloczynu \\





% ####################
\Work{
  N. M. Matwiejew \\
  ,,Metody całkowania równana różniczkowych zwyczajnych'',
  \cite{MatwiejewMetodyCalkowaniaRownanRozniczkowychZwyczajnych82} }


\CenterTB{Uwagi}

\start \Str{15} Równanie (10) zostało bardzo elegancko wyprowadzone,
przy założeniu, że~funkcja uwikłana dana równaniem (9) spełnia
równanie różniczkowe (1), nie odpowiada to jednak na pytanie czy jeśli
funkcja $y( x )$ spełnia równanie (10), to spełnia też interesujące
nas równanie wyjściowe. Dowód tego twierdzenia jest następujący.
Załóżmy, że $\Phi_{ y }' \neq 0$, tak by było zapewnione istnienie
funkcji uwikłanej. Jeżeli teraz spełnione jest równanie (10), to można
je przekształcić do postaci:
\begin{equation}
  \lable{Matwiejew-01}
  -\fr{ \Phi_{ x }' }{ \Phi_{ y }' } = f( x, y ).
\end{equation}
Lewa strona tej równości jest równa pochodnej funkcji uwikłanej
$y( x )$, określonej wzorem (9). \\
\start \Str{16} \Dok \\
\start \Str{31} \Dok


\CenterTB{Błędy}
\begin{center}
  \begin{tabular}{|c|c|c|c|c|}
    \hline
    & \multicolumn{2}{c|}{} & & \\
    Strona & \multicolumn{2}{c|}{Wiersz} & Jest
                              & Powinno być \\ \cline{2-3}
    & Od góry & Od dołu & & \\
    \hline
    5   & &  9 & Dodzimy & Dowodzimy \\
    5   & &  8 & potkowych & początkowych \\
    5   & &  7 & poąątkowych & początkowych \\
    10  & & 19 & damy & mamy \\
    15  & & & & \\
    % & & & & \\
    % & & & & \\
    \hline
  \end{tabular}
\end{center}
\noi \\
\StrWg{15}{13}
\Jest w~sensie ustępu \\
\Powin w~sensie zdefiniowanym w~ustępie \\
\StrWd{20}{2} \\
\Jest i~nie ma rozwiązania określonego w~tym samym przedziale
nie~identycznego z~rozwiązaniem $y = y( x )$ chociażby w~jednym
punkcie przedziału $\abso{ x - x_{ 0 } } \leq h$ różnym
od~punktu $x = x_{ 0 }$. \\
\Powin i~nie istnieje inne rozwiązanie określone w~przedziale
$\abso{ x - x_{ 0 } } \leq h_{ 1 } \leq h$ które nie byłoby równe
rozwiązaniu $y = y( x )$ w~każdym punkcie przedziału
$\abso{ x - x_{ 0 } } \leq h_{ 1 }$. \\

\vspace{\spaceTwo}










% ####################################################################
% ####################################################################
\bibliographystyle{plalpha} \bibliography{LibMathInfo}{}


% ############################

% Koniec dokumentu
\end{document}
